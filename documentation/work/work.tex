% !TeX spellcheck = russian-aot


% Полезные ресурсы: 
% https://www.tablesgenerator.com/ -- генератор таблиц
% Для цитирования используйте \fcite{}

\documentclass[a4paper,12pt]{extarticle}
\usepackage{headers}
\makeatletter
\def\@maketitle{%
	\newpage
	УДК \@udk
	\begin{flushright}
		\begin{tabular}[t]{r}%
			\@author
		\end{tabular}\par%
	\end{flushright}
	\vskip 1em%
	\begin{center}%
		\let \footnote \thanks
		{\LARGE \textbf{\@title} \par}%
		\vskip 1.5em%
		{\large
			\lineskip .5em%
		}
		%\vskip 1em%
		%{\large \@date}%
	\end{center}%
	\par
	\vskip 1.5em}
\makeatother
\addbibresource{work.bib}
\udk{343.140.02:004.891.2}
\author{\nameauthor{Винокуров Михаил Андреевич}{студент 2 курса}{специальности <<Безопасность информационных технологий в правоохранительной сфере>>}{vinokurov.m.a@edu.mirea.ru}   
\nameteacher{Ксенофонтов Николай Валерьевич}{старший преподаватель КБ-4}{}{ksenofontov@mirea.ru}}
\title{Нейронная сеть для анализа российской арбитражной судебной практики}
\makeindex
\begin{document}
\singlespacing{\maketitle}
\section*{Введение}
Целью данной работы является создание нейронной сети для анализа
российской арбитражной судебной практики по банкротствам. В наши дни
количество арбитражных дел неуклонно растет, поэтому вопрос полуавтоматизации
стоит весьма остро. К примеру, в сервисе <<Картотека арбитражных
дел>>\fcite{arbitration-library}, по состоянию на 2024-06-07T13:30+03:00, в
картотеке за период от 2024-06-06 до 2024-06-07 зарегистрировано 1808 
банкротных дел.

Однако, стоит отметить, что до машинноисполняемого
права ещё далеко ввиду динамичности и системности права. Cудебное усмотрение в арбитражном процессе предполагает применение, в
том числе, доктрин, часть из которых и сформировалась благодаря дискреционным
полномочиям суда: доктрина нечистых рук, поднятая корпоративная вуаль,
эстоппель (запрет на противоречивые действия), frustration of contract, contra proferentem (который нашёл явное
отражение в пункте 11 Постановления Пленума Высшего Арбитражного Суда РФ от 14
марта 2014 г. N 16 <<О свободе договора и ее пределах>>). Добавим
ещё механизмы вроде contra legem и судебного нормотворчества ad hoc, насчёт
которых до сих пор идут споры, и получим ад для искуственного интеллекта.
Стоит отметить, что судебное усмотрение для арбитражного является критически
важным для выполнения задач, описанных в статье 2 АПК РФ.

А особенно важной, по мнению автора, является задача, описанная в пункте 3
статьи 2 АПК РФ:

\begin{description}
    \item \textbf{Справедливое публичное судебное разбирательство в разумный срок
        независимым и беспристрасным судом.}\fcite{arbitr1}
\end{description}

Выделенная задача опирается на этико-нравственные оценочные категории, к
примеру, справедливость.
\section{Техническое требование}
\subsection{Функциональные требования}
\begin{itemize}
    \item[-] Создание модели, обученной на судебной практике. 
    \item[-] Создание системы, позволяющей по запросу находить наиболее ревелатные дела.
    \item[-] Борьба с <<галлюцинациями>>, вызванными особенностями модели.
\end{itemize}

\subsection{Нефункциональные требования} 
\begin{itemize}
    \item[-] Реализация на ЯП Java.
    \item[-] Использование bindings for CUDA для ускорения обучения модели.
    \item[-] CLI интерфейс.
\end{itemize}

 
\subsection{Особенности реализации}
\begin{itemize}
    \item[-] Снято требование <<Размер используемого хипа \ul{не должен} превышать 500мб >> пункта 3 части 3 Требований по выполнению курсовой работы из-за особенностей проекта.
\end{itemize}


\subsection{Необходимые ресурсы и технологии}
\begin{itemize}
    \item[-] Java\fcite{required-source}
    \item[-] Word2Vec\fcite{deeplearning4j}
    \item[-] Java bindings for CUDA
    \item[-] Судебная прак тика по банкротствам. (могут возникнуть проблемы из-за
        подпункта 1 пункта 5 статьи 15 Федерального закона от 22.12.2008
        \textnumero\ 262-ФЗ (ред. от 14.07.2022) <<Об обеспечении доступа к
        информации о деятельности судов в Российской Федерации>>)
\end{itemize}
 
\section{Описание выполнения задания}
Начнём с изучения произвольного дела Арбитражного суда РФ. Рассмотрим дело
А82-8071/2024. На данный момент там есть одно-единственное определение о принятии к производству заявления о признании гражданина банкротом
и назначении судебного заседания. Как и все определения в картотеке, оно
находится в PDF. Поскольку PDF является бинарным форматом, его надо перевести в более простой для
обработки формат. Для этого воспользуемся классом PDFParser, исходный код
которого расположится ниже:
\lstinputlisting[language=Java, caption=Класс PDFParser]{../../practice_analysis/src/main/java/ru/inferno_geek/arbitrage_analysis/PDFParser.java}
После перевода PDF в более простой формат мы можем воспользоваться классом
Vectorizer для получения текстовых векторов для дальнейшего обучения нейронных сетей.
\lstinputlisting[language=Java, caption=Класс Vectorizer]{../../practice_analysis/src/main/java/ru/inferno_geek/arbitrage_analysis/Vectorizer.java}
\section{Благодарности}
\begin{description}
    \item Ксенофонтов Николай Валерьевич за методологию BDSM без партнёра.
    \item Котилевец Игорь Денисович за мотивацию и поддержку и полезные
        материалы.
        \end{description}
\section{GitHub}
\url{https://github.com/Infernogeek1/court-practice-analysis-java}
        \printbibliography[heading=bibintoc, title={\centering{Список использованных источников}}]
\end{document}
