% !TeX spellcheck = russian-aot


% Полезные ресурсы: 
% https://www.tablesgenerator.com/ -- генератор таблиц
% Для цитирования используйте \fcite{}

\documentclass[a4paper,12pt]{extarticle}
\usepackage{amsmath}
\usepackage{amssymb} 
\usepackage[utf8]{inputenc}
\usepackage[T2A]{fontenc}
\usepackage{csquotes}
\usepackage[english, russian]{babel}
\usepackage{enumitem}
\usepackage{indentfirst} % красная строка
\usepackage{setspace} % межстрочный интервал
\setstretch{1.2}  
\usepackage{soul}
\usepackage{titling}% изменение титула
\usepackage{fancyhdr}
% \usepackage[]{hyperref}
% \usepackage{chemfig}
\usepackage[parentracker=true,
backend=biber,
hyperref=false,
bibencoding=utf8,
style=gost-footnote,
language=russian,
autolang=other,
citestyle=gost-footnote,
defernumbers=true,
bibstyle=gost-numeric,
sorting=none,
]{biblatex} % ЦИТАТЫ


\setlist{nosep}
\usepackage[lmargin=2.0cm, tmargin=2.0cm, bmargin=2.0cm,rmargin=2.0cm]{geometry}
\usepackage{graphicx}
\usepackage{float}

\usepackage[labelsep=period,justification=RaggedLeft]{caption}
\addto\captionsrussian{\renewcommand{\figurename}{Рисунок}}
% \captionsetup[figure]{labelfont+=it}

\newcommand{\fcite}[1]{\footcite{#1}}
\newcommand{\prim}[1]{#1\ensuremath{^1}\kern-\scriptspace}

\newcommand{\nameauthor}[4]{\textbf{#1} \\
	\textit{#2}\\ 
	\textit{#3} \\
	\textit{Института кибербезопасности и цифровых технологий}\\
	\textit{РТУ МИРЭА} \\
	\makeatletter\textit{#4}\makeatother\\
	\vspace{0.25em}\\}
	
\newcommand{\nameteacher}[4]{
	\textbf{Научный руководитель:  #1}\\ 
	\textit{#2}\\ 
	\textit{#3}\\
	\textit{Института кибербезопасности и цифровых технологий}\\
	\textit{РТУ МИРЭА} \\
	\makeatletter\textit{#4}\makeatother\\}
	
\makeatletter\newcommand{\@udk}{Something went wrong}
\newcommand{\udk}[1]{\renewcommand{\@udk}{#1}}\makeatother

\makeatletter
\def\@maketitle{%
	\newpage
	УДК \@udk
	\begin{flushright}
		\begin{tabular}[t]{r}%
			\@author
		\end{tabular}\par%
	\end{flushright}
	\vskip 1em%
	\begin{center}%
		\let \footnote \thanks
		{\LARGE \textbf{\@title} \par}%
		\vskip 1.5em%
		{\large
			\lineskip .5em%
		}
		%\vskip 1em%
		%{\large \@date}%
	\end{center}%
	\par
	\vskip 1.5em}
\makeatother
\usepackage{siunitx}
\addbibresource{template.bib}
\udk{343.140.02:004.891.2}
\author{\nameauthor{Винокуров Михаил Андреевич}{студент 2 курса}{специальности <<Безопасность информационных технологий в правоохранительной сфере>>}{vinokurov.m.a@edu.mirea.ru}   
\nameteacher{Ксенофонтов Николай Валерьевич}{старший преподаватель КБ-4}{}{ksenofontov@mirea.ru}}
\title{Нейронная сеть для анализа российской уголовной судебной практики: техническое задание}
\begin{document}
\singlespacing{\maketitle}

\section{Описание проекта}
В данном проекте будет рассматриваться создание нейронной сети для анализа российской уголовной судебной практики по преступлениям из раздела IX <<Преступления против общественной безопасности и общественного порядка>>, главы 28 <<Преступления в сфере компьютерной информации>>. 
\section{Функциональные требования}
\begin{itemize}
    \item[-] Создание модели, обученной на судебной практике. 
    \item[-] Создание системы, позволяющей по запросу находить наиболее ревелатные дела.
    \item[-] Борьба с <<галлюцинациями>>, вызванными особенностями модели.
\end{itemize}

\section{Нефункциональные требования}
\begin{itemize}
    \item[-] Реализация на ЯП Java.
    \item[-] Использование bindings for CUDA для ускорения обучения модели.
    \item[-] CLI интерфейс.
\end{itemize}


\section{Особенности реализации}
\begin{itemize}
    \item[-] Снято требование <<Размер используемого хипа \ul{не должен} превышать 500мб >> пункта 3 части 3 Требований по выполнению курсовой работы из-за особенностей проекта.
\end{itemize}


\section{Необходимые ресурсы и технологии}
\begin{itemize}
    \item[-] Java
    \item[-] Генеративные нейронные сети
    \item[-] Java bindings for CUDA
    \item[-] Судебная практика по разделу IX главе 28 УК РФ. (могут возникнуть проблемы из-за подпункта 1 пункта 5 статьи 15 Федерального закона от 22.12.2008 \textnumero 262-ФЗ (ред. от 14.07.2022) <<Об обеспечении доступа к информации о деятельности судов в Российской Федерации>>
\end{itemize}
\section{Планируемый срок разработки}
3 месяца
\section{Команда разработчиков}
Винокуров М.А
\end{document}

	
